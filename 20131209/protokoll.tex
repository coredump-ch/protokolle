%%% PREAMBLE %%%

% Strict mode

\RequirePackage[l2tabu, orthodox]{nag} % Warn when using deprecated constructs

% Document class and packages

\documentclass[10pt,a4paper,parskip,fleqn]{scrartcl}
\usepackage[a4paper,vmargin={30mm},hmargin={30mm}]{geometry} % Page margins
\usepackage[ngerman]{babel} % New German hyphenation (multilingual support)
\usepackage[utf8x]{inputenc} % Unicode support
\usepackage{graphicx} % Graphics support
\usepackage{enumitem}

% Font configuration

\usepackage[sc]{mathpazo} % Use palatino font
\usepackage[T1]{fontenc} % Use correct font encoding
\usepackage[babel=true]{microtype} % Micro-typographic optimizations
\addtokomafont{disposition}{\rmfamily} % Set palatino as heading font

% Easier enumerations

\newcommand{\ol}{\begin{enumerate}[itemsep=-0.3em,topsep=-0.3em]}
\newcommand{\lo}{\end{enumerate}}
\newcommand{\ul}{\begin{itemize}[itemsep=-0.3em,topsep=-0.3em]}
\newcommand{\lu}{\end{itemize}}
\newcommand{\li}{\item}


%%% TITLEPAGE %%%

\title{Protokoll Vorstandssitzung}
\subject{coredump rapperswil}
\date{9. Dezember 2013}


%%% HEADER / FOOTER %%%

\usepackage{fancyhdr} % Fancy headers
\usepackage{lastpage} % Page numbering

% Capture title and author
\makeatletter
\let\Date\@date
\makeatother

% Fancy headers configuration
\pagestyle{fancy}
\fancyhead{} % Clear all header fields
\fancyhead[LO,LE]{\bfseries Protokoll}
\fancyhead[RO,RE]{\bfseries \Date{}}
\fancyfoot{} % Clear all footer fields
\fancyfoot[CO,CE]{Seite \thepage}
\renewcommand{\headrulewidth}{0.3pt}
\renewcommand{\footrulewidth}{0pt}

% Better footer
\cfoot{Seite \thepage\ von \pageref{LastPage}}


%%% MAIN DOCUMENT %%%

\begin{document}

\begin{titlepage}

	\maketitle
	\thispagestyle{empty} % Don't start page numbers on this page

	\vfill

	\begin{description}
		\item[Beginn] 09. Dezember 2013, 21:00 Uhr
		\item[Ende] 09. Dezember 2013, 22:30 Uhr
		\item[Ort] Vereinslokal
		\item[Protokoll] Raphael Nestler
		\item[Anwesend] Danilo Bargen, Josua Schmid, Raphael Nestler
	\end{description}

\end{titlepage}


\section{Finanzen}

\subsection{Monatliche Kosten}

Monatliche Einnahmen:

\ul
	\li Sponsoren: 140.-
	\li Mitglieder: 210.-
\lu

Monatliche Ausgaben:

\ul
	\li Miete: 320.-
\lu

Differenz: \textbf{350 - 320 = 30 CHF}

\subsection{Reserven}

Als Reserven möchten wir im Minimum immer 1000 CHF (3 Monatsmieten aufgerundet)
auf dem Konto haben.

\subsection{Budget 2014}

Ausstehende Mitgliederbeiträge bis Ende Jahr: 280.-

Geschätztes Vermögen Ende Jahr: 1800.-

Für 2014 Budgetierbar: 800.-

\subsection{Ausgaben-Policy}

Was bezahlen wir?

\ul
	\li Grundbestand Elektronik-Bauteile
	\li Gemeinsame Entwicklungskits (Arduino, Raspberry Pi, ...)
	\li Alles was im Raum bleibt
	\li Wenn coredump etwas bezahlen soll, Vorstand per E-Mail oder direkt
	anfragen. Bei Kostenübernahme muss das Projekt zugänglich gemacht werden (zB
	im Coredump-Github) und grob dokumentiert sein (mind. ein README file).
\lu

\subsection{Investitionen}

Fiverr-Auftrag: \textit{It's time to dump core}. Kosten 5 USD.

Einstimmig angenommen!!

\section{Coole Projektideen}

\subsection{Salami!!! und sonstige Würste und Bier}

Temperaturschrank mit geregelter Temperatur und Feuchtigkeit (Sensi-Sensoren?)

Sensi Bier?

\subsection{Intelligentes Pissoir}

Pinkeln und Aussage ob genug getrunken, Alkoholpegel, etc. (Sensi-Chemiesensor)


\subsection{3D-Drucker}

Gut für andere Projekte. Jemand müsste mal eine Kostenaufstellung für einen
RepRap machen.

\section{Marketing}

\subsection{HSR}

Nadeldrucker in HSR aufstellen. Webseite, auf welcher man beliebige Texte
eingeben und ausdrucken kann. Webcam lädt dann Foto von Print hoch.

Erweiterungsmöglichkeit: Twitterbot reagiert auf gewisse Tweets, druckt sie aus,
und schickt Bild von Ausdruck als Reply zurück.

\subsection{Nerd-Night im Bären}

Projekte im Bären präsentieren?

\subsection{Ferienpass Rappi-Jona}

Eletronik-Nachmittag für Schüler. Ideen:

\ul
	\li LED Streifen mit Potis.
	\li Programmieren lernen.
\lu

Danilo meldet sich beim Ferienpass, Wuschel macht Elektronik.

\subsection{Berufsschulen, Kantis}

Mögliche Zusammenarbeit? Josua fragt an.

\section{Networking}

\subsection{Andere ''Coredumps`` aka Hackerspaces besuchen}

zB Ruum42 in St. Gallen oder CCC-ZH.


\vspace{1cm}

Ende der Sitzung: 22:30 Uhr

\end{document}
