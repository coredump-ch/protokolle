%%% PREAMBLE %%%

% Strict mode

\RequirePackage[l2tabu, orthodox]{nag} % Warn when using deprecated constructs

% Document class and packages

\documentclass[10pt,a4paper,parskip,fleqn]{scrartcl}
\usepackage[a4paper,vmargin={30mm},hmargin={30mm}]{geometry} % Page margins
\usepackage[ngerman]{babel} % New German hyphenation (multilingual support)
\usepackage[utf8x]{inputenc} % Unicode support
\usepackage{graphicx} % Graphics support
\usepackage{enumitem}

% Font configuration

\usepackage[sc]{mathpazo} % Use palatino font
\usepackage[T1]{fontenc} % Use correct font encoding
\usepackage[babel=true]{microtype} % Micro-typographic optimizations
\addtokomafont{disposition}{\rmfamily} % Set palatino as heading font

% Easier enumerations

\newcommand{\ol}{\begin{enumerate}[itemsep=-0.3em,topsep=-0.3em]}
\newcommand{\lo}{\end{enumerate}}
\newcommand{\ul}{\begin{itemize}[itemsep=-0.3em,topsep=-0.3em]}
\newcommand{\lu}{\end{itemize}}
\newcommand{\li}{\item}


%%% TITLEPAGE %%%

\title{Protokoll Vorstandssitzung}
\subject{coredump rapperswil}
\date{23. Juni 2014}


%%% HEADER / FOOTER %%%

\usepackage{fancyhdr} % Fancy headers
\usepackage{lastpage} % Page numbering

% Capture title and author
\makeatletter
\let\Date\@date
\makeatother

% Fancy headers configuration
\pagestyle{fancy}
\fancyhead{} % Clear all header fields
\fancyhead[LO,LE]{\bfseries Protokoll}
\fancyhead[RO,RE]{\bfseries \Date{}}
\fancyfoot{} % Clear all footer fields
\fancyfoot[CO,CE]{Seite \thepage}
\renewcommand{\headrulewidth}{0.3pt}
\renewcommand{\footrulewidth}{0pt}

% Better footer
\cfoot{Seite \thepage\ von \pageref{LastPage}}


%%% MAIN DOCUMENT %%%

\begin{document}

\begin{titlepage}

	\maketitle
	\thispagestyle{empty} % Don't start page numbers on this page

	\vfill

	\begin{description}
		\item[Beginn] 23. Juni 2014, 21:00 Uhr
		\item[Ende] 23. Juni 2014, 22:00 Uhr
		\item[Ort] Vereinslokal
		\item[Protokoll] Raphael Nestler
		\item[Anwesend] Danilo Bargen, Josua Schmid, Raphael Nestler
	\end{description}

\end{titlepage}

\section{Stand Projekte}


\section{Coole Projektideen}

\subsection{Wassertemperatursensor}

Wassertemperatur im Obersee messen und zur Verfügung stellen.

Kommunikation?
\ul
\li	SMS
\lu

Energie?
\ul
\li	Wellenbewegung
\li Wind
\li Solar
\li Temperaturdifferenz
\lu

\subsection{Drink-Mix-Roboter}

\texttt{http://bit.ly/1opd1Bf}


\subsection{Codegolf}


\section{Mitgliedschaft CCC}

5.- pro Mitglied pro Jahr. Wäre zahlbar.

\textbf{Mail an alle Mitglieder.}

\section{Flip-Dot Display}

Tauschgeschäft mit CCC. Wir bieten Bier, wir können das Display ausleihen.


\section{Mitgliederbeiträge}

Einige hier ungenannte Mitglieder müssen noch ihren Mitgliederbeitrag
überweisen.


\section{Marketing}


\subsection{Ferienpass Rappi-Jona}

\ul
\li	Programmieren: 14.10. und 17.10.
\li Löten:	13.10. und 16.10.
\lu

Material: 10 Lötsets, 8 RaspberryPi

Zusammenfassungstext: Gemeinsam, siehe github

E-Mail an Mitglieder: Danilo

Antwort an Ferienpass: Danilo

\subsection{Berufsschulen, Kantis}

Raphi fragt in Berufsschule Uster via Flurin nach.

Möglichkeiten:
\ul
\li Vortrag mit Demonstration
\lu


\section{Networking}

\subsection{Andere ''Coredumps`` aka Hackerspaces besuchen}

zB Ruum42 in St. Gallen oder CCC-ZH.

CCC-ZH wäre mal interessant

\vspace{1cm}

Ende der Sitzung: 22:26 Uhr

\end{document}
